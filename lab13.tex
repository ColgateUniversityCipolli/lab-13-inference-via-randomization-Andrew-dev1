\documentclass{article}\usepackage[]{graphicx}\usepackage[]{xcolor}
% maxwidth is the original width if it is less than linewidth
% otherwise use linewidth (to make sure the graphics do not exceed the margin)
\makeatletter
\def\maxwidth{ %
  \ifdim\Gin@nat@width>\linewidth
    \linewidth
  \else
    \Gin@nat@width
  \fi
}
\makeatother

\definecolor{fgcolor}{rgb}{0.345, 0.345, 0.345}
\newcommand{\hlnum}[1]{\textcolor[rgb]{0.686,0.059,0.569}{#1}}%
\newcommand{\hlsng}[1]{\textcolor[rgb]{0.192,0.494,0.8}{#1}}%
\newcommand{\hlcom}[1]{\textcolor[rgb]{0.678,0.584,0.686}{\textit{#1}}}%
\newcommand{\hlopt}[1]{\textcolor[rgb]{0,0,0}{#1}}%
\newcommand{\hldef}[1]{\textcolor[rgb]{0.345,0.345,0.345}{#1}}%
\newcommand{\hlkwa}[1]{\textcolor[rgb]{0.161,0.373,0.58}{\textbf{#1}}}%
\newcommand{\hlkwb}[1]{\textcolor[rgb]{0.69,0.353,0.396}{#1}}%
\newcommand{\hlkwc}[1]{\textcolor[rgb]{0.333,0.667,0.333}{#1}}%
\newcommand{\hlkwd}[1]{\textcolor[rgb]{0.737,0.353,0.396}{\textbf{#1}}}%
\let\hlipl\hlkwb

\usepackage{framed}
\makeatletter
\newenvironment{kframe}{%
 \def\at@end@of@kframe{}%
 \ifinner\ifhmode%
  \def\at@end@of@kframe{\end{minipage}}%
  \begin{minipage}{\columnwidth}%
 \fi\fi%
 \def\FrameCommand##1{\hskip\@totalleftmargin \hskip-\fboxsep
 \colorbox{shadecolor}{##1}\hskip-\fboxsep
     % There is no \\@totalrightmargin, so:
     \hskip-\linewidth \hskip-\@totalleftmargin \hskip\columnwidth}%
 \MakeFramed {\advance\hsize-\width
   \@totalleftmargin\z@ \linewidth\hsize
   \@setminipage}}%
 {\par\unskip\endMakeFramed%
 \at@end@of@kframe}
\makeatother

\definecolor{shadecolor}{rgb}{.97, .97, .97}
\definecolor{messagecolor}{rgb}{0, 0, 0}
\definecolor{warningcolor}{rgb}{1, 0, 1}
\definecolor{errorcolor}{rgb}{1, 0, 0}
\newenvironment{knitrout}{}{} % an empty environment to be redefined in TeX

\usepackage{alltt}
\usepackage[margin=1.0in]{geometry} % To set margins
\usepackage{amsmath}  % This allows me to use the align functionality.
                      % If you find yourself trying to replicate
                      % something you found online, ensure you're
                      % loading the necessary packages!
\usepackage{amsfonts} % Math font
\usepackage{fancyvrb}
\usepackage{hyperref} % For including hyperlinks
\usepackage[shortlabels]{enumitem}% For enumerated lists with labels specified
                                  % We had to run tlmgr_install("enumitem") in R
\usepackage{float}    % For telling R where to put a table/figure
\usepackage{natbib}        %For the bibliography
\bibliographystyle{apalike}%For the bibliography
\IfFileExists{upquote.sty}{\usepackage{upquote}}{}
\begin{document}


\begin{enumerate}
%%%%%%%%%%%%%%%%%%%%%%%%%%%%%%%%%%%%%%%%%%%%%%%%%%%%%%%%%%%%%%%%%%%%%%%%%%%%%%%%
%%%%%%%%%%%%%%%%%%%%%%%%%%%%%%%%%%%%%%%%%%%%%%%%%%%%%%%%%%%%%%%%%%%%%%%%%%%%%%%%
% Question 1
%%%%%%%%%%%%%%%%%%%%%%%%%%%%%%%%%%%%%%%%%%%%%%%%%%%%%%%%%%%%%%%%%%%%%%%%%%%%%%%%
%%%%%%%%%%%%%%%%%%%%%%%%%%%%%%%%%%%%%%%%%%%%%%%%%%%%%%%%%%%%%%%%%%%%%%%%%%%%%%%%
\item When conducting the work of Lab 11, we conducted the test that uses the
Central Limit Theorem even though the sample size was ``small" (i.e., $n<30$).
It turns out, that how ``far off" the $t$-test is can be computed using
a first-order Edgeworth approximation for the error. Below, we will do this 
for the the further observations.
\begin{enumerate}
  \item \cite{Boos00} note that 
  \begin{align*}
    P(T \leq t) \approx F_Z(t) + \underbrace{\frac{\text{skew}}{\sqrt{n}} \frac{(2t^2+1)}{6} f_Z(t)}_{\textrm{error}},
  \end{align*}
  where $f_Z(\cdot)$ and $F_Z(\cdot)$ are the Gaussian PDF and CDF and skew is the
  skewness of the data. What is the potential error in the computation of the 
  $p$-value when testing $H_0: \mu_X=0; H_a: \mu_X<0$ using the zebra finch further data? 
\begin{knitrout}\scriptsize
\definecolor{shadecolor}{rgb}{0.969, 0.969, 0.969}\color{fgcolor}\begin{kframe}
\begin{alltt}
\hlcom{## 1a}
\hldef{zebra.finches.dat} \hlkwb{<-} \hlkwd{read_csv}\hldef{(}\hlsng{"zebrafinches.csv"}\hldef{)}

\hldef{further.vec} \hlkwb{<-} \hldef{zebra.finches.dat}\hlopt{$}\hldef{further}
\hldef{skew} \hlkwb{<-} \hlkwd{skewness}\hldef{(further.vec)}
\hldef{n} \hlkwb{<-} \hlkwd{length}\hldef{(further.vec)}

\hldef{further.test} \hlkwb{<-} \hlkwd{t.test}\hldef{(}\hlkwc{x}\hldef{=further.vec,} \hlkwc{mu} \hldef{=} \hlnum{0}\hldef{,} \hlkwc{alternative} \hldef{=} \hlsng{"less"}\hldef{)}
\hldef{t.far} \hlkwb{<-} \hldef{further.test}\hlopt{$}\hldef{statistic} \hlcom{#finding the t value }
\hldef{fz} \hlkwb{<-} \hlkwd{dnorm}\hldef{(t.far)}
\hldef{Fz} \hlkwb{<-} \hlkwd{pnorm}\hldef{(t.far)}
\hldef{(error} \hlkwb{=} \hldef{(skew} \hlopt{/} \hlkwd{sqrt}\hldef{(n))} \hlopt{*} \hldef{((}\hlnum{2} \hlopt{*} \hldef{(t.far}\hlopt{^}\hlnum{2}\hldef{)} \hlopt{+} \hlnum{1}\hldef{)} \hlopt{/} \hlnum{6}\hldef{)} \hlopt{*} \hldef{fz)}
\end{alltt}
\begin{verbatim}
##             t 
## -1.226006e-13
\end{verbatim}
\begin{alltt}
\hldef{prob} \hlkwb{<-} \hldef{error} \hlopt{+} \hldef{Fz}
\end{alltt}
\end{kframe}
\end{knitrout}
  \item Compute the error for $t$ statistics from -10 to 10 and plot a line
  that shows the error across $t$. Continue to use the skewness and 
  the sample size for the zebra finch further data.
\begin{knitrout}\scriptsize
\definecolor{shadecolor}{rgb}{0.969, 0.969, 0.969}\color{fgcolor}\begin{kframe}
\begin{alltt}
\hldef{t.vecs} \hlkwb{<-} \hlkwd{seq}\hldef{(}\hlopt{-}\hlnum{10}\hldef{,}\hlnum{10}\hldef{,}\hlnum{0.1}\hldef{)}
\hldef{results} \hlkwb{<-} \hlkwd{tibble}\hldef{(}
  \hlkwc{t} \hldef{= t.vecs,}
  \hlkwc{fz} \hldef{=} \hlkwd{dnorm}\hldef{(t.vecs),}
  \hlkwc{error} \hldef{= (skew} \hlopt{/} \hlkwd{sqrt}\hldef{(n))} \hlopt{*} \hldef{((}\hlnum{2} \hlopt{*} \hldef{(t.vecs}\hlopt{^}\hlnum{2}\hldef{)} \hlopt{+} \hlnum{1}\hldef{)} \hlopt{/} \hlnum{6}\hldef{)} \hlopt{*} \hlkwd{dnorm}\hldef{(t.vecs)}
\hldef{)}

\hldef{error.plot} \hlkwb{<-} \hlkwd{ggplot}\hldef{(results)} \hlopt{+}
  \hlkwd{theme_bw}\hldef{()}\hlopt{+}
  \hlkwd{geom_line}\hldef{(}\hlkwd{aes}\hldef{(}\hlkwc{x} \hldef{= t,} \hlkwc{y} \hldef{= error))}\hlopt{+}
  \hlkwd{labs}\hldef{(}
    \hlkwc{x} \hldef{=} \hlsng{"T values"}\hldef{,}
    \hlkwc{title} \hldef{=} \hlsng{"Edgeworth Approximation Error"}
  \hldef{)}
\end{alltt}
\end{kframe}
\end{knitrout}

\begin{figure}[H]
\begin{center}
\begin{knitrout}
\definecolor{shadecolor}{rgb}{0.969, 0.969, 0.969}\color{fgcolor}
\includegraphics[width=\maxwidth]{figure/unnamed-chunk-4-1} 
\end{knitrout}
\caption{Edgeworth Error across span of t values}
\label{plot1} 
\end{center}
\end{figure}

  \item Suppose we wanted to have a tail probability within 10\% of the desired
  $\alpha=0.05$. Recall we did a left-tailed test using the further data.
  How large of a sample size would we need? That is, we need
  to solve the error formula equal to 10\% of the desired left-tail probability:
  \[0.10 \alpha  \stackrel{set}{=} \underbrace{\frac{\text{skew}}{\sqrt{n}} \frac{(2t^2+1)}{6} f_Z(t)}_{\textrm{error}},\]
  which yields
  \[ n = \left(\frac{\text{skew}}{6(0.10\alpha)} (2t^2 + 1) f_Z(t)\right)^2.\]

\textbf{Answer: }We would need a sample size of at least 521 for a left-tailed test.
\begin{knitrout}\scriptsize
\definecolor{shadecolor}{rgb}{0.969, 0.969, 0.969}\color{fgcolor}\begin{kframe}
\begin{alltt}
\hldef{alpha} \hlkwb{=} \hlnum{0.05}
\hldef{t.val} \hlkwb{<-} \hlkwd{qnorm}\hldef{(alpha)}
\hldef{pdf} \hlkwb{<-} \hlkwd{dnorm}\hldef{(t.val)}
\hldef{(target.size} \hlkwb{=} \hldef{((skew} \hlopt{/} \hldef{(}\hlnum{6}\hlopt{*}\hldef{(}\hlnum{0.1}\hlopt{*}\hldef{alpha)))} \hlopt{*} \hldef{(}\hlnum{2} \hlopt{*} \hldef{(t.val}\hlopt{^}\hlnum{2}\hldef{)} \hlopt{+} \hlnum{1}\hldef{)} \hlopt{*} \hldef{pdf)}\hlopt{^}\hlnum{2}\hldef{)}
\end{alltt}
\begin{verbatim}
## [1] 520.8876
\end{verbatim}
\end{kframe}
\end{knitrout}

\end{enumerate}
%%%%%%%%%%%%%%%%%%%%%%%%%%%%%%%%%%%%%%%%%%%%%%%%%%%%%%%%%%%%%%%%%%%%%%%%%%%%%%%%
%%%%%%%%%%%%%%%%%%%%%%%%%%%%%%%%%%%%%%%%%%%%%%%%%%%%%%%%%%%%%%%%%%%%%%%%%%%%%%%%
% Question 2
%%%%%%%%%%%%%%%%%%%%%%%%%%%%%%%%%%%%%%%%%%%%%%%%%%%%%%%%%%%%%%%%%%%%%%%%%%%%%%%%
%%%%%%%%%%%%%%%%%%%%%%%%%%%%%%%%%%%%%%%%%%%%%%%%%%%%%%%%%%%%%%%%%%%%%%%%%%%%%%%%
\item Complete the following steps to revisit the analyses from lab 11 using the
bootstrap procedure.
\begin{enumerate}
\item Now, consider the zebra finch data. We do not know the generating distributions
for the closer, further, and difference data, so perform resampling to approximate the 
sampling distribution of the $T$ statistic:
  \[T = \frac{\bar{x}_r - 0}{s/\sqrt{n}},\]
  where $\bar{x}_r$ is the mean computed on the r$^{th}$ resample and $s$ is the
  sample standard deviation from the original samples. At the end, create an
  object called \texttt{resamples.null.closer}, for example, and store the 
  resamples shifted to ensure they are consistent with the null hypotheses at the average 
  (i.e., here ensure the shifted resamples are 0 on average, corresponding
  to $t=0$, for each case). 
\\ \textbf{Answer: } All of the shifted means of the t values are so small that they are practically 0.

\begin{knitrout}\scriptsize
\definecolor{shadecolor}{rgb}{0.969, 0.969, 0.969}\color{fgcolor}\begin{kframe}
\begin{alltt}
\hlcom{###########################}
\hlcom{#  Part 2:Boot strapping  #}
\hlcom{###########################}
\hldef{R} \hlkwb{<-} \hlnum{10000}
\hldef{mu0} \hlkwb{=} \hlnum{0}
\hldef{fur.sd} \hlkwb{<-} \hlkwd{sd}\hldef{(zebra.finches.dat}\hlopt{$}\hldef{further)}
\hldef{close.sd} \hlkwb{<-} \hlkwd{sd}\hldef{(zebra.finches.dat}\hlopt{$}\hldef{closer)}
\hldef{diff.sd} \hlkwb{<-} \hlkwd{sd}\hldef{(zebra.finches.dat}\hlopt{$}\hldef{diff)}

\hldef{resample} \hlkwb{<-} \hlkwd{tibble}\hldef{(}\hlkwc{close.t}\hldef{=}\hlkwd{numeric}\hldef{(R),}
                   \hlkwc{far.t}\hldef{=}\hlkwd{numeric}\hldef{(R),}
                   \hlkwc{diff.t}\hldef{=}\hlkwd{numeric}\hldef{(R),}
                   \hlkwc{close.mean} \hldef{=}\hlkwd{numeric}\hldef{(R),}
                   \hlkwc{far.mean} \hldef{=}\hlkwd{numeric}\hldef{(R),}
                   \hlkwc{diff.mean} \hldef{=}\hlkwd{numeric}\hldef{(R)}
\hldef{)}
\hldef{resamples.shifted} \hlkwb{<-} \hlkwd{tibble}\hldef{()}


\hlkwa{for}\hldef{(i} \hlkwa{in} \hlnum{1}\hlopt{:}\hldef{R)\{}
  \hldef{close.sample} \hlkwb{<-} \hlkwd{sample}\hldef{(}\hlkwc{x}\hldef{=zebra.finches.dat}\hlopt{$}\hldef{closer,}
                         \hlkwc{size}\hldef{=n,}
                         \hlkwc{replace}\hldef{=T)}
  \hldef{far.sample} \hlkwb{<-} \hlkwd{sample}\hldef{(}\hlkwc{x}\hldef{=zebra.finches.dat}\hlopt{$}\hldef{further,}
                       \hlkwc{size}\hldef{=n,}
                       \hlkwc{replace}\hldef{=T)}
  \hldef{diff.sample} \hlkwb{<-} \hlkwd{sample}\hldef{(}\hlkwc{x}\hldef{=zebra.finches.dat}\hlopt{$}\hldef{diff,}
                        \hlkwc{size}\hldef{=n,}
                        \hlkwc{replace}\hldef{=T)}

  \hldef{resample}\hlopt{$}\hldef{close.t[i]} \hlkwb{=} \hldef{(}\hlkwd{mean}\hldef{(close.sample)}\hlopt{-}\hldef{mu0)}\hlopt{/}\hldef{(close.sd}\hlopt{/}\hlkwd{sqrt}\hldef{(n))}
  \hldef{resample}\hlopt{$}\hldef{far.t[i]} \hlkwb{=} \hldef{(}\hlkwd{mean}\hldef{(far.sample)}\hlopt{-}\hldef{mu0)}\hlopt{/}\hldef{(fur.sd}\hlopt{/}\hlkwd{sqrt}\hldef{(n))}
  \hldef{resample}\hlopt{$}\hldef{diff.t[i]} \hlkwb{=} \hldef{(}\hlkwd{mean}\hldef{(diff.sample)}\hlopt{-}\hldef{mu0)}\hlopt{/}\hldef{(diff.sd}\hlopt{/}\hlkwd{sqrt}\hldef{(n))}
  \hldef{resample}\hlopt{$}\hldef{close.mean[i]} \hlkwb{=} \hlkwd{mean}\hldef{(close.sample)}
  \hldef{resample}\hlopt{$}\hldef{far.mean[i]} \hlkwb{=} \hlkwd{mean}\hldef{(far.sample)}
  \hldef{resample}\hlopt{$}\hldef{diff.mean[i]} \hlkwb{=} \hlkwd{mean}\hldef{(diff.sample)}

\hldef{\}}

\hlcom{## shifting the means}
\hldef{close.change} \hlkwb{<-} \hlkwd{mean}\hldef{(resample}\hlopt{$}\hldef{close.t)}
\hldef{far.change} \hlkwb{<-} \hlkwd{mean}\hldef{(resample}\hlopt{$}\hldef{far.t)}
\hldef{diff.change} \hlkwb{<-} \hlkwd{mean}\hldef{(resample}\hlopt{$}\hldef{diff.t)}

\hldef{resamples.shifted} \hlkwb{<-} \hldef{resample |>}
  \hlkwd{mutate}\hldef{(}\hlkwc{close.shifted} \hldef{= close.t} \hlopt{-} \hldef{close.change)  |>}
  \hlkwd{mutate}\hldef{(}\hlkwc{far.shifted} \hldef{= far.t} \hlopt{-} \hldef{far.change) |>}
  \hlkwd{mutate}\hldef{(}\hlkwc{diff.shifted} \hldef{= diff.t} \hlopt{-} \hldef{diff.change) |>}
  \hlkwd{select}\hldef{(}\hlkwd{c}\hldef{(close.shifted, far.shifted, diff.shifted ))}
\hldef{(}\hlkwd{mean}\hldef{(resamples.shifted}\hlopt{$}\hldef{close.shifted))}
\end{alltt}
\begin{verbatim}
## [1] 1.171951e-16
\end{verbatim}
\begin{alltt}
\hldef{(}\hlkwd{mean}\hldef{(resamples.shifted}\hlopt{$}\hldef{far.shifted))}
\end{alltt}
\begin{verbatim}
## [1] 1.382103e-16
\end{verbatim}
\begin{alltt}
\hldef{(}\hlkwd{mean}\hldef{(resamples.shifted}\hlopt{$}\hldef{diff.shifted))}
\end{alltt}
\begin{verbatim}
## [1] 3.648193e-16
\end{verbatim}
\end{kframe}
\end{knitrout}
  \item Compute the bootstrap $p$-value for each test using the shifted resamples.
  How do these compare to the $t$-test $p$-values?

\textbf{Answer: }Table \ref{table1} below shows that the comparison of the p-values. While t-tests did produce nonzero p-values, they were so small that they are essentially 0, which is what the bootstrap test shows for each case.

\begin{knitrout}\scriptsize
\definecolor{shadecolor}{rgb}{0.969, 0.969, 0.969}\color{fgcolor}\begin{kframe}
\begin{alltt}
\hlcom{# Bootstrap P-Value}
\hldef{boot.p.closer} \hlkwb{<-} \hlkwd{mean}\hldef{(resamples.shifted}\hlopt{$}\hldef{close.shifted} \hlopt{>=} \hldef{close.change)}
\hldef{boot.p.far} \hlkwb{<-} \hlkwd{mean}\hldef{(resamples.shifted}\hlopt{$}\hldef{far.shifted} \hlopt{<=} \hldef{far.change)}
\hldef{low} \hlkwb{=} \hlopt{-}\hldef{diff.change}
\hldef{high} \hlkwb{=} \hldef{diff.change}
\hldef{p.low} \hlkwb{<-} \hlkwd{mean}\hldef{(resamples.shifted}\hlopt{$}\hldef{diff.shifted} \hlopt{<=} \hldef{low)}
\hldef{p.high} \hlkwb{<-} \hlkwd{mean}\hldef{(resamples.shifted}\hlopt{$}\hldef{diff.shifted} \hlopt{>=} \hldef{high)}
\hldef{boot.p.diff} \hlkwb{<-} \hldef{p.low} \hlopt{+}\hldef{p.high}

\hlcom{# T-Test P-Value}
\hldef{test.close} \hlkwb{<-} \hlkwd{t.test}\hldef{(zebra.finches.dat}\hlopt{$}\hldef{closer,}
                     \hlkwc{mu} \hldef{=} \hlnum{0}\hldef{,}
                     \hlkwc{alternative} \hldef{=} \hlsng{"greater"}\hldef{)}
\hldef{close.ttest.p} \hlkwb{<-} \hldef{test.close}\hlopt{$}\hldef{p.value}

\hldef{test.far} \hlkwb{<-} \hlkwd{t.test}\hldef{(zebra.finches.dat}\hlopt{$}\hldef{further,}
                     \hlkwc{mu} \hldef{=} \hlnum{0}\hldef{,}
                     \hlkwc{alternative} \hldef{=} \hlsng{"less"}\hldef{)}
\hldef{far.ttest.p} \hlkwb{<-} \hldef{test.far}\hlopt{$}\hldef{p.value}

\hldef{test.diff} \hlkwb{<-} \hlkwd{t.test}\hldef{(zebra.finches.dat}\hlopt{$}\hldef{diff,}
                   \hlkwc{mu} \hldef{=} \hlnum{0}\hldef{,}
                   \hlkwc{alternative} \hldef{=} \hlsng{"two.sided"}\hldef{)}
\hldef{diff.ttest.p} \hlkwb{<-} \hldef{test.diff}\hlopt{$}\hldef{p.value}

\hldef{comparisons.p} \hlkwb{<-} \hlkwd{tibble}\hldef{(}\hlsng{" "}\hldef{=} \hlkwd{c}\hldef{(}\hlsng{"close"}\hldef{,} \hlsng{"far"}\hldef{,} \hlsng{"diff"}\hldef{),} \hlkwc{Bootstrapped} \hldef{=}
                        \hlkwd{c}\hldef{(boot.p.closer, boot.p.far, boot.p.diff),}
                      \hlkwc{t.test} \hldef{=} \hlkwd{c}\hldef{(close.ttest.p, far.ttest.p, diff.ttest.p))}
\end{alltt}
\end{kframe}
\end{knitrout}

\begin{table}[H]
\centering
\begin{tabular}{lrr}
  \hline
 Data & Bootstrapped & T test \\ 
  \hline
 closer & 0.00 & 0.00 \\ 
 further & 0.00 & 0.00 \\ 
 difference & 0.00 & 0.00 \\ 
   \hline
\end{tabular} \caption{Comparison of p-values between bootstrapping and T test} \label{table1}
\end{table}


    \item What is the 5$^{th}$ percentile of the shifted resamples under the null hypothesis? 

\textbf{Answer: }Table \ref{table2} below shows that the comparison of the 5$^{th}$ percentile of the shifted resamples: closer = -1.54, further = -1.70, and difference = -1.61. They approximate the T-test, which is -1.71.
\begin{knitrout}\scriptsize
\definecolor{shadecolor}{rgb}{0.969, 0.969, 0.969}\color{fgcolor}\begin{kframe}
\begin{alltt}
\hlcom{## Part c}
\hldef{close.5th} \hlkwb{<-} \hlkwd{quantile}\hldef{(resamples.shifted}\hlopt{$}\hldef{close.shifted,} \hlnum{0.05}\hldef{)}
\hldef{far.5th} \hlkwb{<-} \hlkwd{quantile}\hldef{(resamples.shifted}\hlopt{$}\hldef{far.shifted,} \hlnum{0.05}\hldef{)}
\hldef{diff.5th} \hlkwb{<-} \hlkwd{quantile}\hldef{(resamples.shifted}\hlopt{$}\hldef{diff.shifted,} \hlnum{0.05}\hldef{)}

\hldef{close.percentile.t} \hlkwb{<-} \hlkwd{qt}\hldef{(}\hlnum{0.05}\hldef{,} \hlkwc{df} \hldef{= n}\hlopt{-}\hlnum{1}\hldef{)}
\hldef{far.percentile.t} \hlkwb{<-} \hlkwd{qt}\hldef{(}\hlnum{0.05}\hldef{,} \hlkwc{df} \hldef{= n}\hlopt{-}\hlnum{1}\hldef{)}
\hldef{diff.percentile.t} \hlkwb{<-} \hlkwd{qt}\hldef{(}\hlnum{0.05}\hldef{,} \hlkwc{df} \hldef{= n}\hlopt{-}\hlnum{1}\hldef{)}

\hldef{comparisons.percentiles} \hlkwb{<-} \hlkwd{tibble}\hldef{(}\hlsng{" "}\hldef{=} \hlkwd{c}\hldef{(}\hlsng{"close"}\hldef{,} \hlsng{"far"}\hldef{,} \hlsng{"diff"}\hldef{),} \hlkwc{sampled} \hldef{=}
                         \hlkwd{c}\hldef{(close.5th, far.5th, diff.5th),}
                       \hlkwc{t.val} \hldef{=}\hlkwd{c}\hldef{(close.percentile.t, far.percentile.t,}
                                 \hldef{diff.percentile.t))}
\end{alltt}
\end{kframe}
\end{knitrout}

\begin{table}[H]
\centering
\begin{tabular}{lrr}
  \hline
Data & Bootstrapped & T-test \\ 
  \hline
  Closer & -1.54 & -1.71 \\ 
  Further & -1.70 & -1.71 \\ 
  Difference & -1.61 & -1.71 \\ 
   \hline
\end{tabular} \caption{Comparison of the 5th percentile between bootstrapping and T test} \label{table2}
\end{table}

  Note this value approximates $t_{0.05, n-1}$. Compare these values in each case.
  \item Compute the bootstrap confidence intervals using the resamples. How do these compare to the $t$-test confidence intervals?
 
 \textbf{Answer: }Table \ref{table3} below shows that the comparison of the confidence intervals of the shifted resamples with those of the T-tests. Respectively for the closer, further, and difference cases, bootstrapping produced 95\% confidence intervals of (0.12, 0.18), (-0.26, -0.17), and (0.27, 0.42) while the T-tests produced CIs of (0.12, 0.20), (-0.26, -0.15), and (0.27, 0.45). They are pretty similar to each other for each case 

\begin{knitrout}\scriptsize
\definecolor{shadecolor}{rgb}{0.969, 0.969, 0.969}\color{fgcolor}\begin{kframe}
\begin{alltt}
\hlcom{## Part d}
\hldef{lower.close} \hlkwb{<-} \hlkwd{quantile}\hldef{(resample}\hlopt{$}\hldef{close.mean,} \hlnum{0.025}\hldef{)}
\hldef{upper.close} \hlkwb{<-} \hlkwd{quantile}\hldef{(resample}\hlopt{$}\hldef{close.mean,} \hlnum{0.925}\hldef{)}
\hldef{Bootstrap.CI.close} \hlkwb{<-} \hlkwd{c}\hldef{(lower.close, upper.close)}

\hldef{lower.far} \hlkwb{<-} \hlkwd{quantile}\hldef{(resample}\hlopt{$}\hldef{far.mean,} \hlnum{0.025}\hldef{)}
\hldef{upper.far} \hlkwb{<-} \hlkwd{quantile}\hldef{(resample}\hlopt{$}\hldef{far.mean,} \hlnum{0.925}\hldef{)}
\hldef{Bootstrap.CI.far} \hlkwb{<-} \hlkwd{c}\hldef{(lower.far, upper.far)}

\hldef{lower.diff} \hlkwb{<-} \hlkwd{quantile}\hldef{(resample}\hlopt{$}\hldef{diff.mean,} \hlnum{0.025}\hldef{)}
\hldef{upper.diff} \hlkwb{<-} \hlkwd{quantile}\hldef{(resample}\hlopt{$}\hldef{diff.mean,} \hlnum{0.925}\hldef{)}
\hldef{Bootstrap.CI.diff} \hlkwb{<-} \hlkwd{c}\hldef{(lower.diff, upper.diff)}

\hlcom{# t tests confidence intervals}
\hldef{test.close} \hlkwb{<-} \hlkwd{t.test}\hldef{(zebra.finches.dat}\hlopt{$}\hldef{closer,}
                    \hlkwc{mu} \hldef{=} \hlnum{0}\hldef{,} \hlkwc{conf.level} \hldef{=} \hlnum{0.95}\hldef{,}
                    \hlkwc{alternative} \hldef{=} \hlsng{"two.sided"}\hldef{)}
\hldef{close.CI} \hlkwb{<-} \hldef{test.close}\hlopt{$}\hldef{conf.int}

\hldef{test.far} \hlkwb{<-} \hlkwd{t.test}\hldef{(zebra.finches.dat}\hlopt{$}\hldef{further,}
                    \hlkwc{mu} \hldef{=} \hlnum{0}\hldef{,} \hlkwc{conf.level} \hldef{=} \hlnum{0.95}\hldef{,}
                    \hlkwc{alternative} \hldef{=} \hlsng{"two.sided"}\hldef{)}
\hldef{far.CI} \hlkwb{<-} \hldef{test.far}\hlopt{$}\hldef{conf.int}

\hldef{test.diff} \hlkwb{<-} \hlkwd{t.test}\hldef{(zebra.finches.dat}\hlopt{$}\hldef{diff,}
                    \hlkwc{mu} \hldef{=} \hlnum{0}\hldef{,} \hlkwc{conf.level} \hldef{=} \hlnum{0.95}\hldef{,}
                    \hlkwc{alternative} \hldef{=} \hlsng{"two.sided"}\hldef{)}
\hldef{diff.CI} \hlkwb{<-} \hldef{test.diff}\hlopt{$}\hldef{conf.int}

\hldef{comparisons.CI} \hlkwb{<-} \hlkwd{tibble}\hldef{(}
  \hlsng{"Condition"} \hldef{=} \hlkwd{c}\hldef{(}\hlsng{"close"}\hldef{,} \hlsng{"far"}\hldef{,} \hlsng{"diff"}\hldef{),}
  \hlsng{"Bootstrapped Lower"} \hldef{=} \hlkwd{c}\hldef{(lower.close, lower.far, lower.diff),}
  \hlsng{"Bootstrapped Upper"} \hldef{=} \hlkwd{c}\hldef{(upper.close, upper.far, upper.diff),}
  \hlsng{"t-test Lower"} \hldef{=} \hlkwd{c}\hldef{(close.CI[}\hlnum{1}\hldef{], far.CI[}\hlnum{1}\hldef{], diff.CI[}\hlnum{1}\hldef{]),}
  \hlsng{"t-test Upper"} \hldef{=} \hlkwd{c}\hldef{(close.CI[}\hlnum{2}\hldef{], far.CI[}\hlnum{2}\hldef{], diff.CI[}\hlnum{2}\hldef{]))}
\end{alltt}
\end{kframe}
\end{knitrout}
\begin{table}[H]
\centering
\begin{tabular}{lrrrr}
  \hline
 Data & Bootstrapped Lower & Bootstrapped Upper & t-test Lower & t-test Upper \\ 
  \hline
 close & 0.12 & 0.18 & 0.12 & 0.20 \\ 
 far & -0.26 & -0.17 & -0.26 & -0.15 \\ 
 diff & 0.27 & 0.42 & 0.27 & 0.45 \\ 
   \hline
\end{tabular} \caption{Comparison of the Confidence Interval between bootstrapping and T test, separated into lower and upper bounds} \label{table3}
\end{table}


\end{enumerate}
%%%%%%%%%%%%%%%%%%%%%%%%%%%%%%%%%%%%%%%%%%%%%%%%%%%%%%%%%%%%%%%%%%%%%%%%%%%%%%%%
%%%%%%%%%%%%%%%%%%%%%%%%%%%%%%%%%%%%%%%%%%%%%%%%%%%%%%%%%%%%%%%%%%%%%%%%%%%%%%%%
% Question 3
%%%%%%%%%%%%%%%%%%%%%%%%%%%%%%%%%%%%%%%%%%%%%%%%%%%%%%%%%%%%%%%%%%%%%%%%%%%%%%%%
%%%%%%%%%%%%%%%%%%%%%%%%%%%%%%%%%%%%%%%%%%%%%%%%%%%%%%%%%%%%%%%%%%%%%%%%%%%%%%%%
\item Complete the following steps to revisit the analyses from lab 11 using the
randomization procedure.
\begin{enumerate}
\item Now, consider the zebra finch data. We do not know the generating distributions
for the closer, further, and difference data, so perform the randomization procedure
\begin{knitrout}\scriptsize
\definecolor{shadecolor}{rgb}{0.969, 0.969, 0.969}\color{fgcolor}\begin{kframe}
\begin{alltt}
\hldef{mu0} \hlkwb{<-} \hlnum{0}
\hldef{R} \hlkwb{<-} \hlnum{10000}
\hldef{randomized} \hlkwb{<-} \hlkwd{tibble}\hldef{(}\hlkwc{close.means} \hldef{=} \hlkwd{numeric}\hldef{(R),} \hlkwc{far.means} \hldef{=} \hlkwd{numeric}\hldef{(R),}
                  \hlkwc{diff.means} \hldef{=} \hlkwd{numeric}\hldef{(R))}

\hldef{shifted.x.close} \hlkwb{<-} \hldef{zebra.finches.dat}\hlopt{$}\hldef{closer} \hlopt{-} \hldef{mu0}
\hldef{shifted.x.far} \hlkwb{<-} \hldef{zebra.finches.dat}\hlopt{$}\hldef{further} \hlopt{-} \hldef{mu0}
\hldef{shifted.x.diff} \hlkwb{<-} \hldef{zebra.finches.dat}\hlopt{$}\hldef{diff} \hlopt{-} \hldef{mu0}

\hlkwa{for}\hldef{(i} \hlkwa{in} \hlnum{1}\hlopt{:}\hldef{R)\{}
  \hldef{curr.rand1} \hlkwb{<-} \hldef{shifted.x.close} \hlopt{*}
    \hlkwd{sample}\hldef{(}\hlkwc{x} \hldef{=} \hlkwd{c}\hldef{(}\hlopt{-}\hlnum{1}\hldef{,} \hlnum{1}\hldef{),}
           \hlkwc{size} \hldef{=} \hlkwd{length}\hldef{(shifted.x.close),}
           \hlkwc{replace} \hldef{= T)}
  \hldef{curr.rand2} \hlkwb{<-} \hldef{shifted.x.far} \hlopt{*}
    \hlkwd{sample}\hldef{(}\hlkwc{x} \hldef{=} \hlkwd{c}\hldef{(}\hlopt{-}\hlnum{1}\hldef{,} \hlnum{1}\hldef{),}
           \hlkwc{size} \hldef{=} \hlkwd{length}\hldef{(shifted.x.far),}
           \hlkwc{replace} \hldef{= T)}
  \hldef{curr.rand3} \hlkwb{<-} \hldef{shifted.x.diff} \hlopt{*}
    \hlkwd{sample}\hldef{(}\hlkwc{x} \hldef{=} \hlkwd{c}\hldef{(}\hlopt{-}\hlnum{1}\hldef{,} \hlnum{1}\hldef{),}
           \hlkwc{size} \hldef{=} \hlkwd{length}\hldef{(shifted.x.diff),}
           \hlkwc{replace} \hldef{= T)}

  \hldef{randomized}\hlopt{$}\hldef{close.means[i]} \hlkwb{<-} \hlkwd{mean}\hldef{(curr.rand1)}
  \hldef{randomized}\hlopt{$}\hldef{far.means[i]} \hlkwb{<-} \hlkwd{mean}\hldef{(curr.rand2)}
  \hldef{randomized}\hlopt{$}\hldef{diff.means[i]} \hlkwb{<-} \hlkwd{mean}\hldef{(curr.rand3)}
\hldef{\}}

\hldef{randomized} \hlkwb{<-} \hldef{randomized |>}
  \hlkwd{mutate}\hldef{(}\hlkwc{close.means} \hldef{= close.means} \hlopt{+} \hldef{mu0,}
         \hlkwc{far.means} \hldef{= far.means} \hlopt{+} \hldef{mu0,}
         \hlkwc{diff.means} \hldef{= diff.means} \hlopt{+} \hldef{mu0 )} \hlcom{# shifting data back}
\end{alltt}
\end{kframe}
\end{knitrout}

  \item Compute the randomization test $p$-value for each test.
  
  \textbf{Answer: }The p value is 0 for all three cases, so the close, further, and difference data are in the tailed regions of their respective null distributions. The closer data is in its right-tailed zone, further in its left-tailed zone, and difference in both tailed zones, so they all reject the null hypothesis.
\begin{knitrout}\scriptsize
\definecolor{shadecolor}{rgb}{0.969, 0.969, 0.969}\color{fgcolor}\begin{kframe}
\begin{alltt}
\hlcom{############################################}
\hlcom{## 3b}
\hlcom{## found observed data }
\hldef{observed.close.mean} \hlkwb{<-} \hlkwd{mean}\hldef{(zebra.finches.dat}\hlopt{$}\hldef{closer} \hlopt{-}\hldef{mu0)}
\hldef{observed.far.mean} \hlkwb{<-} \hlkwd{mean}\hldef{(zebra.finches.dat}\hlopt{$}\hldef{further} \hlopt{-}\hldef{mu0)}
\hldef{observed.diff.mean} \hlkwb{<-} \hlkwd{mean}\hldef{(zebra.finches.dat}\hlopt{$}\hldef{diff} \hlopt{-}\hldef{mu0)}

\hlcom{## p-values for each value type }
\hldef{rand.p.close} \hlkwb{<-} \hlkwd{mean}\hldef{(randomized}\hlopt{$}\hldef{close.means} \hlopt{>=} \hldef{observed.close.mean)}
\hldef{rand.p.far} \hlkwb{<-} \hlkwd{mean}\hldef{(randomized}\hlopt{$}\hldef{far.means} \hlopt{<=} \hldef{observed.far.mean)}

\hldef{(delta} \hlkwb{<-} \hlkwd{abs}\hldef{(}\hlkwd{mean}\hldef{(zebra.finches.dat}\hlopt{$}\hldef{diff)} \hlopt{-} \hldef{mu0))}
\end{alltt}
\begin{verbatim}
## [1] 0.3589475
\end{verbatim}
\begin{alltt}
\hldef{(low} \hlkwb{<-} \hldef{mu0} \hlopt{-} \hldef{delta)} \hlcom{# mirror}
\end{alltt}
\begin{verbatim}
## [1] -0.3589475
\end{verbatim}
\begin{alltt}
\hldef{(high}\hlkwb{<-} \hldef{mu0} \hlopt{+} \hldef{delta)}   \hlcom{# xbar}
\end{alltt}
\begin{verbatim}
## [1] 0.3589475
\end{verbatim}
\begin{alltt}
\hldef{rand.p.diff} \hlkwb{<-} \hlkwd{mean}\hldef{(randomized}\hlopt{$}\hldef{diff.means} \hlopt{<=} \hldef{low)} \hlopt{+}
  \hlkwd{mean}\hldef{(randomized}\hlopt{$}\hldef{diff.means} \hlopt{>=} \hldef{high)}

\hlkwd{c}\hldef{(rand.p.close,rand.p.far,rand.p.diff)}
\end{alltt}
\begin{verbatim}
## [1] 0 0 0
\end{verbatim}
\end{kframe}
\end{knitrout}

  \item Compute the randomization confidence interval by iterating over values of $\mu_0$.\\
  \textbf{Hint:} You can ``search" for the lower bound from $Q_1$ and subtracting by 0.0001, 
  and the upper bound using $Q_3$ and increasing by 0.0001. You will continue until you find 
  the first value for which the two-sided $p$-value is greater than or equal to 0.05.
  
\textbf{Answer: } We have to numerically compute the confidence interval of a randomization test, so we slowly decreased the mean until the p-value hit 0.05 to find the lower bound and slowly increased the mean until the same condition to find the upper bound for each case. Our final result showed a confidence interval for the closer data as (0.1189, 0.1941), for the further data (-0.2627, -0.1427), and for the difference data (0.2689, 0.4489). 
\begin{knitrout}\scriptsize
\definecolor{shadecolor}{rgb}{0.969, 0.969, 0.969}\color{fgcolor}\begin{kframe}
\begin{alltt}
\hlcom{############################################}
\hlcom{## 3c}
\hlcom{## creating randomized Confidence Intervals }

\hldef{R} \hlkwb{<-} \hlnum{1000}
\hldef{mu0.iterate} \hlkwb{<-} \hlnum{0.0001}
\hldef{starting.point} \hlkwb{<-} \hlkwd{mean}\hldef{(zebra.finches.dat}\hlopt{$}\hldef{closer)}
\hldef{mu.lower.close} \hlkwb{<-} \hldef{starting.point}

\hlkwa{repeat}\hldef{\{}
  \hldef{rand} \hlkwb{<-} \hlkwd{tibble}\hldef{(}\hlkwc{xbars} \hldef{=} \hlkwd{rep}\hldef{(}\hlnum{NA}\hldef{, R))}

  \hlcom{# PREPROCESSING: shift the data to be mean 0 under H0}
  \hldef{x.shift} \hlkwb{<-} \hldef{zebra.finches.dat}\hlopt{$}\hldef{closer} \hlopt{-} \hldef{mu.lower.close}
  \hlcom{# RANDOMIZE / SHUFFLE}
  \hlkwa{for}\hldef{(i} \hlkwa{in} \hlnum{1}\hlopt{:}\hldef{R)\{}
    \hldef{curr.rand} \hlkwb{<-} \hldef{x.shift} \hlopt{*}
      \hlkwd{sample}\hldef{(}\hlkwc{x} \hldef{=} \hlkwd{c}\hldef{(}\hlopt{-}\hlnum{1}\hldef{,} \hlnum{1}\hldef{),}
             \hlkwc{size} \hldef{=} \hlkwd{length}\hldef{(x.shift),}
             \hlkwc{replace} \hldef{= T)}

    \hldef{rand}\hlopt{$}\hldef{xbars[i]} \hlkwb{<-} \hlkwd{mean}\hldef{(curr.rand)}
  \hldef{\}}

  \hldef{rand} \hlkwb{<-} \hldef{rand |>}
    \hlkwd{mutate}\hldef{(}\hlkwc{xbars} \hldef{= xbars} \hlopt{+} \hldef{mu.lower.close)} \hlcom{# shifting back}

  \hlcom{# p-value }
  \hldef{delta} \hlkwb{<-} \hlkwd{abs}\hldef{(}\hlkwd{mean}\hldef{(zebra.finches.dat}\hlopt{$}\hldef{closer)} \hlopt{-} \hldef{mu.lower.close)}
  \hldef{low} \hlkwb{<-} \hldef{mu.lower.close} \hlopt{-} \hldef{delta} \hlcom{# mirror}
  \hldef{high}\hlkwb{<-} \hldef{mu.lower.close} \hlopt{+} \hldef{delta}   \hlcom{# xbar}
  \hldef{p.val} \hlkwb{<-} \hlkwd{mean}\hldef{(rand}\hlopt{$}\hldef{xbars} \hlopt{<=} \hldef{low)} \hlopt{+}
      \hlkwd{mean}\hldef{(rand}\hlopt{$}\hldef{xbars} \hlopt{>=} \hldef{high)}


  \hlkwa{if}\hldef{(p.val} \hlopt{<} \hlnum{0.05}\hldef{)\{}
    \hlkwa{break}
  \hldef{\}}\hlkwa{else}\hldef{\{}
    \hldef{mu.lower.close} \hlkwb{<-} \hldef{mu.lower.close} \hlopt{-} \hldef{mu0.iterate}
  \hldef{\}}
\hldef{\}}


\hldef{mu.upper.close} \hlkwb{<-} \hldef{starting.point}
\hlkwa{repeat}\hldef{\{}
  \hldef{rand} \hlkwb{<-} \hlkwd{tibble}\hldef{(}\hlkwc{xbars} \hldef{=} \hlkwd{rep}\hldef{(}\hlnum{NA}\hldef{, R))}

  \hlcom{# PREPROCESSING: shift the data to be mean 0 under H0}
  \hldef{x.shift} \hlkwb{<-} \hldef{zebra.finches.dat}\hlopt{$}\hldef{closer} \hlopt{-} \hldef{mu.upper.close}
  \hlcom{# RANDOMIZE / SHUFFLE}
  \hlkwa{for}\hldef{(i} \hlkwa{in} \hlnum{1}\hlopt{:}\hldef{R)\{}
    \hldef{curr.rand} \hlkwb{<-} \hldef{x.shift} \hlopt{*}
      \hlkwd{sample}\hldef{(}\hlkwc{x} \hldef{=} \hlkwd{c}\hldef{(}\hlopt{-}\hlnum{1}\hldef{,} \hlnum{1}\hldef{),}
             \hlkwc{size} \hldef{=} \hlkwd{length}\hldef{(x.shift),}
             \hlkwc{replace} \hldef{= T)}

    \hldef{rand}\hlopt{$}\hldef{xbars[i]} \hlkwb{<-} \hlkwd{mean}\hldef{(curr.rand)}
  \hldef{\}}

  \hldef{rand} \hlkwb{<-} \hldef{rand |>}
    \hlkwd{mutate}\hldef{(}\hlkwc{xbars} \hldef{= xbars} \hlopt{+} \hldef{mu.upper.close)} \hlcom{# shifting back}

  \hlcom{# p-value }
  \hldef{delta} \hlkwb{<-} \hlkwd{abs}\hldef{(}\hlkwd{mean}\hldef{(zebra.finches.dat}\hlopt{$}\hldef{closer)} \hlopt{-} \hldef{mu.upper.close)}
  \hldef{(low} \hlkwb{<-} \hldef{mu.upper.close} \hlopt{-} \hldef{delta)} \hlcom{# mirror}
  \hldef{(high}\hlkwb{<-} \hldef{mu.upper.close} \hlopt{+} \hldef{delta)}   \hlcom{# xbar}
  \hldef{(p.val} \hlkwb{<-} \hlkwd{mean}\hldef{(rand}\hlopt{$}\hldef{xbars} \hlopt{<=} \hldef{low)} \hlopt{+}
      \hlkwd{mean}\hldef{(rand}\hlopt{$}\hldef{xbars} \hlopt{>=} \hldef{high))}

  \hlkwa{if}\hldef{(p.val} \hlopt{<} \hlnum{0.05}\hldef{)\{}
    \hlkwa{break}
  \hldef{\}}\hlkwa{else}\hldef{\{}
    \hldef{mu.upper.close} \hlkwb{<-} \hldef{mu.upper.close} \hlopt{+} \hldef{mu0.iterate}
  \hldef{\}}
\hldef{\}}


\hlcom{## CI for further }
\hldef{starting.point} \hlkwb{<-} \hlkwd{mean}\hldef{(zebra.finches.dat}\hlopt{$}\hldef{further)}
\hldef{mu.lower.far} \hlkwb{<-} \hldef{starting.point}

\hlkwa{repeat}\hldef{\{}
  \hldef{rand} \hlkwb{<-} \hlkwd{tibble}\hldef{(}\hlkwc{xbars} \hldef{=} \hlkwd{rep}\hldef{(}\hlnum{NA}\hldef{, R))}

  \hlcom{# PREPROCESSING: shift the data to be mean 0 under H0}
  \hldef{x.shift} \hlkwb{<-} \hldef{zebra.finches.dat}\hlopt{$}\hldef{further} \hlopt{-} \hldef{mu.lower.far}
  \hlcom{# RANDOMIZE / SHUFFLE}
  \hlkwa{for}\hldef{(i} \hlkwa{in} \hlnum{1}\hlopt{:}\hldef{R)\{}
    \hldef{curr.rand} \hlkwb{<-} \hldef{x.shift} \hlopt{*}
      \hlkwd{sample}\hldef{(}\hlkwc{x} \hldef{=} \hlkwd{c}\hldef{(}\hlopt{-}\hlnum{1}\hldef{,} \hlnum{1}\hldef{),}
             \hlkwc{size} \hldef{=} \hlkwd{length}\hldef{(x.shift),}
             \hlkwc{replace} \hldef{= T)}

    \hldef{rand}\hlopt{$}\hldef{xbars[i]} \hlkwb{<-} \hlkwd{mean}\hldef{(curr.rand)}
  \hldef{\}}

  \hldef{rand} \hlkwb{<-} \hldef{rand |>}
    \hlkwd{mutate}\hldef{(}\hlkwc{xbars} \hldef{= xbars} \hlopt{+} \hldef{mu.lower.far)} \hlcom{# shifting back}

  \hlcom{# p-value }
  \hldef{delta} \hlkwb{<-} \hlkwd{abs}\hldef{(}\hlkwd{mean}\hldef{(zebra.finches.dat}\hlopt{$}\hldef{further)} \hlopt{-} \hldef{mu.lower.far)}
  \hldef{low} \hlkwb{<-} \hldef{mu.lower.far} \hlopt{-} \hldef{delta} \hlcom{# mirror}
  \hldef{high}\hlkwb{<-} \hldef{mu.lower.far} \hlopt{+} \hldef{delta}   \hlcom{# xbar}
  \hldef{p.val} \hlkwb{<-} \hlkwd{mean}\hldef{(rand}\hlopt{$}\hldef{xbars} \hlopt{<=} \hldef{low)} \hlopt{+}
      \hlkwd{mean}\hldef{(rand}\hlopt{$}\hldef{xbars} \hlopt{>=} \hldef{high)}


  \hlkwa{if}\hldef{(p.val} \hlopt{<} \hlnum{0.05}\hldef{)\{}
    \hlkwa{break}
  \hldef{\}}\hlkwa{else}\hldef{\{}
    \hldef{mu.lower.far} \hlkwb{<-} \hldef{mu.lower.far} \hlopt{-} \hldef{mu0.iterate}
  \hldef{\}}
\hldef{\}}


\hldef{mu.upper.far} \hlkwb{<-} \hldef{starting.point}
\hlkwa{repeat}\hldef{\{}
  \hldef{rand} \hlkwb{<-} \hlkwd{tibble}\hldef{(}\hlkwc{xbars} \hldef{=} \hlkwd{rep}\hldef{(}\hlnum{NA}\hldef{, R))}

  \hlcom{# PREPROCESSING: shift the data to be mean 0 under H0}
  \hldef{x.shift} \hlkwb{<-} \hldef{zebra.finches.dat}\hlopt{$}\hldef{further} \hlopt{-} \hldef{mu.upper.far}
  \hlcom{# RANDOMIZE / SHUFFLE}
  \hlkwa{for}\hldef{(i} \hlkwa{in} \hlnum{1}\hlopt{:}\hldef{R)\{}
    \hldef{curr.rand} \hlkwb{<-} \hldef{x.shift} \hlopt{*}
      \hlkwd{sample}\hldef{(}\hlkwc{x} \hldef{=} \hlkwd{c}\hldef{(}\hlopt{-}\hlnum{1}\hldef{,} \hlnum{1}\hldef{),}
             \hlkwc{size} \hldef{=} \hlkwd{length}\hldef{(x.shift),}
             \hlkwc{replace} \hldef{= T)}

    \hldef{rand}\hlopt{$}\hldef{xbars[i]} \hlkwb{<-} \hlkwd{mean}\hldef{(curr.rand)}
  \hldef{\}}

  \hldef{rand} \hlkwb{<-} \hldef{rand |>}
    \hlkwd{mutate}\hldef{(}\hlkwc{xbars} \hldef{= xbars} \hlopt{+} \hldef{mu.upper.far)} \hlcom{# shifting back}

  \hlcom{# p-value }
  \hldef{delta} \hlkwb{<-} \hlkwd{abs}\hldef{(}\hlkwd{mean}\hldef{(zebra.finches.dat}\hlopt{$}\hldef{further)} \hlopt{-} \hldef{mu.upper.far)}
  \hldef{low} \hlkwb{<-} \hldef{mu.upper.far} \hlopt{-} \hldef{delta} \hlcom{# mirror}
  \hldef{high}\hlkwb{<-} \hldef{mu.upper.far} \hlopt{+} \hldef{delta}   \hlcom{# xbar}
  \hldef{p.val} \hlkwb{<-} \hlkwd{mean}\hldef{(rand}\hlopt{$}\hldef{xbars} \hlopt{<=} \hldef{low)} \hlopt{+}
    \hlkwd{mean}\hldef{(rand}\hlopt{$}\hldef{xbars} \hlopt{>=} \hldef{high)}

  \hlkwa{if}\hldef{(p.val} \hlopt{<} \hlnum{0.05}\hldef{)\{}
    \hlkwa{break}
  \hldef{\}}\hlkwa{else}\hldef{\{}
    \hldef{mu.upper.far} \hlkwb{<-} \hldef{mu.upper.far} \hlopt{+} \hldef{mu0.iterate}
  \hldef{\}}
\hldef{\}}


\hlcom{## CI for diff}
\hldef{starting.point} \hlkwb{<-} \hlkwd{mean}\hldef{(zebra.finches.dat}\hlopt{$}\hldef{diff)}
\hldef{mu.lower.diff} \hlkwb{<-} \hldef{starting.point}

\hlkwa{repeat}\hldef{\{}
  \hldef{rand} \hlkwb{<-} \hlkwd{tibble}\hldef{(}\hlkwc{xbars} \hldef{=} \hlkwd{rep}\hldef{(}\hlnum{NA}\hldef{, R))}

  \hlcom{# PREPROCESSING: shift the data to be mean 0 under H0}
  \hldef{x.shift} \hlkwb{<-} \hldef{zebra.finches.dat}\hlopt{$}\hldef{diff} \hlopt{-} \hldef{mu.lower.diff}
  \hlcom{# RANDOMIZE / SHUFFLE}
  \hlkwa{for}\hldef{(i} \hlkwa{in} \hlnum{1}\hlopt{:}\hldef{R)\{}
    \hldef{curr.rand} \hlkwb{<-} \hldef{x.shift} \hlopt{*}
      \hlkwd{sample}\hldef{(}\hlkwc{x} \hldef{=} \hlkwd{c}\hldef{(}\hlopt{-}\hlnum{1}\hldef{,} \hlnum{1}\hldef{),}
             \hlkwc{size} \hldef{=} \hlkwd{length}\hldef{(x.shift),}
             \hlkwc{replace} \hldef{= T)}

    \hldef{rand}\hlopt{$}\hldef{xbars[i]} \hlkwb{<-} \hlkwd{mean}\hldef{(curr.rand)}
  \hldef{\}}

  \hldef{rand} \hlkwb{<-} \hldef{rand |>}
    \hlkwd{mutate}\hldef{(}\hlkwc{xbars} \hldef{= xbars} \hlopt{+} \hldef{mu.lower.diff)} \hlcom{# shifting back}

  \hlcom{# p-value }
  \hldef{delta} \hlkwb{<-} \hlkwd{abs}\hldef{(}\hlkwd{mean}\hldef{(zebra.finches.dat}\hlopt{$}\hldef{diff)} \hlopt{-} \hldef{mu.lower.diff)}
  \hldef{low} \hlkwb{<-} \hldef{mu.lower.diff} \hlopt{-} \hldef{delta} \hlcom{# mirror}
  \hldef{high}\hlkwb{<-} \hldef{mu.lower.diff} \hlopt{+} \hldef{delta}   \hlcom{# xbar}
  \hldef{p.val} \hlkwb{<-} \hlkwd{mean}\hldef{(rand}\hlopt{$}\hldef{xbars} \hlopt{<=} \hldef{low)} \hlopt{+}
    \hlkwd{mean}\hldef{(rand}\hlopt{$}\hldef{xbars} \hlopt{>=} \hldef{high)}


  \hlkwa{if}\hldef{(p.val} \hlopt{<} \hlnum{0.05}\hldef{)\{}
    \hlkwa{break}
  \hldef{\}}\hlkwa{else}\hldef{\{}
    \hldef{mu.lower.diff} \hlkwb{<-} \hldef{mu.lower.diff} \hlopt{-} \hldef{mu0.iterate}
  \hldef{\}}
\hldef{\}}


\hldef{mu.upper.diff} \hlkwb{<-} \hldef{starting.point}
\hlkwa{repeat}\hldef{\{}
  \hldef{rand} \hlkwb{<-} \hlkwd{tibble}\hldef{(}\hlkwc{xbars} \hldef{=} \hlkwd{rep}\hldef{(}\hlnum{NA}\hldef{, R))}

  \hlcom{# PREPROCESSING: shift the data to be mean 0 under H0}
  \hldef{x.shift} \hlkwb{<-} \hldef{zebra.finches.dat}\hlopt{$}\hldef{diff} \hlopt{-} \hldef{mu.upper.diff}
  \hlcom{# RANDOMIZE / SHUFFLE}
  \hlkwa{for}\hldef{(i} \hlkwa{in} \hlnum{1}\hlopt{:}\hldef{R)\{}
    \hldef{curr.rand} \hlkwb{<-} \hldef{x.shift} \hlopt{*}
      \hlkwd{sample}\hldef{(}\hlkwc{x} \hldef{=} \hlkwd{c}\hldef{(}\hlopt{-}\hlnum{1}\hldef{,} \hlnum{1}\hldef{),}
             \hlkwc{size} \hldef{=} \hlkwd{length}\hldef{(x.shift),}
             \hlkwc{replace} \hldef{= T)}

    \hldef{rand}\hlopt{$}\hldef{xbars[i]} \hlkwb{<-} \hlkwd{mean}\hldef{(curr.rand)}
  \hldef{\}}

  \hldef{rand} \hlkwb{<-} \hldef{rand |>}
    \hlkwd{mutate}\hldef{(}\hlkwc{xbars} \hldef{= xbars} \hlopt{+} \hldef{mu.upper.diff)} \hlcom{# shifting back}

  \hlcom{# p-value }
  \hldef{delta} \hlkwb{<-} \hlkwd{abs}\hldef{(}\hlkwd{mean}\hldef{(zebra.finches.dat}\hlopt{$}\hldef{diff)} \hlopt{-} \hldef{mu.upper.diff)}
  \hldef{low} \hlkwb{<-} \hldef{mu.upper.diff} \hlopt{-} \hldef{delta} \hlcom{# mirror}
  \hldef{high}\hlkwb{<-} \hldef{mu.upper.diff} \hlopt{+} \hldef{delta}   \hlcom{# xbar}
  \hldef{p.val} \hlkwb{<-} \hlkwd{mean}\hldef{(rand}\hlopt{$}\hldef{xbars} \hlopt{<=} \hldef{low)} \hlopt{+}
    \hlkwd{mean}\hldef{(rand}\hlopt{$}\hldef{xbars} \hlopt{>=} \hldef{high)}

  \hlkwa{if}\hldef{(p.val} \hlopt{<} \hlnum{0.05}\hldef{)\{}
    \hlkwa{break}
  \hldef{\}}\hlkwa{else}\hldef{\{}
    \hldef{mu.upper.diff} \hlkwb{<-} \hldef{mu.upper.diff} \hlopt{+} \hldef{mu0.iterate}
  \hldef{\}}
\hldef{\}}


\hldef{rand.CI} \hlkwb{<-} \hlkwd{tibble}\hldef{(}\hlsng{" "} \hldef{=} \hlkwd{c}\hldef{(}\hlsng{"close"}\hldef{,} \hlsng{"far"}\hldef{,} \hlsng{"diff"}\hldef{),}
                  \hlkwc{lower.limit} \hldef{=} \hlkwd{c}\hldef{(mu.lower.close, mu.lower.far, mu.lower.diff),}
                  \hlkwc{upper.limit} \hldef{=} \hlkwd{c}\hldef{(mu.upper.close, mu.upper.far, mu.upper.diff))}
\end{alltt}
\end{kframe}
\end{knitrout}
\end{enumerate}
%%%%%%%%%%%%%%%%%%%%%%%%%%%%%%%%%%%%%%%%%%%%%%%%%%%%%%%%%%%%%%%%%%%%%%%%%%%%%%%%
%%%%%%%%%%%%%%%%%%%%%%%%%%%%%%%%%%%%%%%%%%%%%%%%%%%%%%%%%%%%%%%%%%%%%%%%%%%%%%%%
% Optional Question
%%%%%%%%%%%%%%%%%%%%%%%%%%%%%%%%%%%%%%%%%%%%%%%%%%%%%%%%%%%%%%%%%%%%%%%%%%%%%%%%
%%%%%%%%%%%%%%%%%%%%%%%%%%%%%%%%%%%%%%%%%%%%%%%%%%%%%%%%%%%%%%%%%%%%%%%%%%%%%%%%
\item \textbf{Optional Challenge:} In this lab, you performed resampling to 
approximate the sampling distribution of the $T$ statistic using
\[T = \frac{\bar{x}_r - 0}{s/\sqrt{n}}.\]
I'm curious whether it is better/worse/similar if we computed the statistics
using the sample standard deviation of the resamples ($s_r$), instead of the 
original sample ($s$)
  \[T = \frac{\bar{x}_r - 0}{s_r/\sqrt{n}}.\]
\begin{enumerate}
  \item Perform a simulation study to evaluate the Type I error for conducting this
hypothesis test both ways.
  \item Using the same test case(s) as part (a), compute bootstrap confidence 
  intervals and assess their coverage -- how often do we `capture' the parameter
of interest?
\end{enumerate}
%%%%%%%%%%%%%%%%%%%%%%%%%%%%%%%%%%%%%%%%%%%%%%%%%%%%%%%%%%%%%%%%%%%%%%%%%%%%%%%%
%%%%%%%%%%%%%%%%%%%%%%%%%%%%%%%%%%%%%%%%%%%%%%%%%%%%%%%%%%%%%%%%%%%%%%%%%%%%%%%%
% End Document
%%%%%%%%%%%%%%%%%%%%%%%%%%%%%%%%%%%%%%%%%%%%%%%%%%%%%%%%%%%%%%%%%%%%%%%%%%%%%%%%
%%%%%%%%%%%%%%%%%%%%%%%%%%%%%%%%%%%%%%%%%%%%%%%%%%%%%%%%%%%%%%%%%%%%%%%%%%%%%%%%
\end{enumerate}
\bibliography{bibliography}
\end{document}

